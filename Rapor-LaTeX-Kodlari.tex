\documentclass[conference]{IEEEtran}
\IEEEoverridecommandlockouts
\usepackage{cite}
\usepackage{amsmath,amssymb,amsfonts}
\usepackage{algorithmic}
\usepackage{graphicx}
\usepackage{textcomp}
\usepackage{xcolor}
\begin{document}

\title{Kocaeli Üniversitesi Bilgisayar Mühendisliği Bölümü Programlama Laboratuvarı I - Proje III\\}

\author{\IEEEauthorblockN{İbrahim İnce}
\IEEEauthorblockA{\textit{Mühendislik Fakültesi} \\
\textit{Bilgisayar Mühendisliği Bölümü}\\
2. Öğretim \\
200202102}}
\maketitle

\section{Özet}
Öncelikli kuyruk veri yapılarının kullanıldığı bu projede, iniş ve kalkış pistleri olmak üzere iki adet uçak pisti olan bir havalimanının uçuş yönetim sistemi hazırlanmıştır.\\

Pistlerdeki uygun boş saatlere göre iniş izni isteyen uçakların izinlerinin 
onaylanması veya reddedilmesi, iniş izni talep eden uçakların öncelikleri göz 
önünde bulundurularak yapılmıştır.

\section{Giriş}
Proje C programlama dili ve Code::Blocks IDE'si kullanılarak hazırlanmıştır. \\

Havalimanına iniş yapacak uçakların iniş pistini kullanım
sırasını içeren öncelikli kuyruk "inis\_pisti\_kullanim\_sirasi" öncelikli 
kuyruğudur. Uçak öncelik ve iniş saati bilgisi dikkate alınarak
hazırlanmıştır.\\

Havalimanından kalkış yapacak uçakların kalkış pistini
kullanım sırasını içeren öncelikli kuyruk "kalkis\_pisti\_kullanim\_sirasi" 
öncelikli kuyruğudur. Uçak öncelik ve kalkış saati bilgisine göre
hazırlanmıştır.\\

\section{Yöntem}
İlk olarak bağlı liste veri yapısı kullanılarak oluşturulacak öncelikli kuyruk 
yapısı için gerekli iki adet "struct" yapısı oluşturulmuştur. Bu "struct" 
yapıları "inis\_pisti\_kullanim\_sirası" ve "kalkis\_pisti\_kullanim\_sirasi" 
şeklinde isimlendirilmiştir. "inis\_pisti\_kullanim\_sirası" isimli struct 
yapısı içerisinde:\\

\begin{itemize}
    \item "int" tipinde; uçakların öncelik ID'sini tutan "o\_id" değişkeni,
    \item Uçakların ID'sini tutan "u\_id" değişkeni,
    \item Uçakların iniş saatini tutan "i\_sa" değişkeni,
    \item Uçakların gecikme sürelerini tutan "gecikme" değişkeni,
    \item "struct" tipinde bir sonraki düğümü gösterecek "*sD" işaretçisi 
    tanımlanmıştır.\\
\end{itemize}

"kalkis\_pisti\_kullanim\_sirasi" isimli struct yapısında ise bu
değişkenlere ek olarak "int" tipinde talep edilen iniş saatini tutan "t\_i\_sa" 
değişkeni ve kalkış saatini tutan "k\_sa" değişkeni tanımlanmıştır.\\

\subsection{Öncelikli Kuyruk Yapısı İçin Gerekli Fonksiyonların Oluşturulması}
Bağlı listeler kullanılarak oluşturulacak olan öncelikli kuyruk yapısı için 
gerekli fonksiyonlar hazırlanmıştır. Bu fonksiyonlar:\\

\begin{itemize}
    \item İstenilen verileri tutan düğümü kuyruğun sonuna ekleyen 4 parametreli 
    "sonaEkle" "void" fonksiyonu,
    \item İstenilen verileri tutan düğümü kuyruğun başına ekleyen 4 parametreli 
    "basaEkle" "void" fonksiyonu,
    \item İstenilen verileri tutan düğümü kuyruğun istenilen yerine ekleyen 5 
    parametreli "arayaEkle" "void" fonksiyonu,
    \item Kuyruğun başındaki düğümü silen "bastanSil" "void" fonksiyonu,
    \item Kuyruğun sonundaki düğümü silen "sondanSil" "void" fonksiyonu,
    \item İstenilen bir sıradaki düğümü silen "aradanSil" "void" fonksiyonu,
    \item Kuyruğun boş veya dolu olduğunu belirten "bosMu" "int" fonksiyonu,
    \item Kuyruktaki düğüm değerlerini ekrana yazdıran "yazdir" "void" 
    fonksiyonudur.\\
\end{itemize}

Bu fonksiyonlara ek olarak "kalkis\_pisti\_kullanim\_sirasi" öncelikli kuyruğu 
için gerekli olan fonksiyonlar da aynı ismin başına "k\_" eki eklenerek 
oluşturulmuştur.\\

\subsection{"input.txt" Dosyasından Verilerin Okunması}
"fopen()" fonksiyonu ile okuma modunda açılan "input.txt" dosyasından "fscanf()"
fonksiyonu ile okuma yapılcaktır. "input.txt" dosyasından okunulması istenilen 
sayısal verilerden önce bulunan "oncelik\_id", "ucak\_id", 
"talep\_edilen\_inis\_saati" kelime toplulukları üç defa art arda "fscanf()" 
fonksiyonu kullanılarak atlanmıştır. Ardından "while()" döngüsü içerisinde 
"fscanf()" fonksiyonu ile sıradaki üç sayı okunup, "main()" fonksiyonu 
içerisinde oluşturulan "int" tipindeki "o\_id", "u\_id" ve "i\_sa" 
değişkenlerine atanmıştır.\\

Havalimanı uçuş yönetim sistemi için gerekli işlemler
yapıldıktan sonra ekranda görüntülenebilmesi için mevcut "while()" döngüsü 
sonuna "system("pause");" ifadesi, bir sonraki veriler okunmadan önce ekrandaki 
yazıların temizlenmesi için de "system("cls");" ifadesi kullanılmıştır.\\

\subsection{İlk Uçak Verisinin Okunması}
İlk veri okunmadan önce "inis\_pisti\_kullanim\_sirasi" öncelikli kuyruğunda 
gerekli sıralama işlemlerinin yapılabilmesi için tüm değerleri "-100" ve "100" 
olan iki adet düğüm "sonaEkle()" fonksiyonu ile art arda öncelikli kuyruğa 
eklenmiştir.  Öncelikli kuyrukta bulunabilecek maksimum düğüm sayısı 26 adettir. 24 
adet uçak verisi ve kontrol amaçlı eklenen 2 adet düğüm dışında ekleme 
yapılamaz. \\

"input.txt" dosyasından ilk uçak verisi okunduktan sonra "bosMu()" 
fonksiyonu ile "inis\_pisti\_kullanim\_sirasi" öncelikli kuyruğunda önceden 
eklenmiş iki adet düğüm dışında başka düğüm olmadığı teyit edilir ve okunan 
veriler "arayaEkle()" fonksiyonu ile halihazırda bulunan iki düğüm arasına 
eklenir.\\

\subsection{Okunan Uçak Verilerinin Öncelikli Kuyrukta Uygun Yerlere Eklenmesi}
"input.txt" dosyasından ilk uçak verisi okunup öncelikli kuyruğa eklendikten 
sonra sıradaki uçak verisi dosyadan okunur. Öncelikli kuyruğun azami 
kapasitesi olan 26 adet düğüme henüz ulaşılmadığı kontrol edildikten sonra 
okunan uçak verisinin "i\_sa" değişkeninde tutulan "talep\_edilen\_inis\_saati" 
değerinin kuyrukta bulunan düğümlerdeki verilerle çakışıp çakışmadığı kontrol 
edilir. Eğer çakışma yoksa "arayaEkle()" fonksiyonu ile yeni uçak verisi 
öncelikli kuyruktaki uygun yere eklenir.\\

Eğer çakışma var ise uçak yeni kontrollere maruz bırakılır. Oluşabilecek üç adet 
durum vardır. Bu durumlar; yeni eklenecek uçağın, kuyruktaki iniş saati çakışan 
uçaktan daha öncelikli olması (ilk durum olsun), kuyruktaki iniş saati çakışan 
uçağın yeni uçaktan daha öncelikli olması (ikinci durum olsun) ve her iki uçağın
da eşit önceliğe sahip olmasıdır (üçüncü durum olsun). Her durum için farklı 
işlemler yapılır.\\

\begin{figure}[htbp]
\includegraphics[width=9cm, height=22cm]{akissema1.png}
\caption{İniş İzni Kontrolü Akış Şeması}
\end{figure}

İlk durumda kuyruktaki mevcut uçağın azami gecikme süresi olan
3 saat ertelenip ertelenmediği kontrol edilir. Eğer azami gecikme süresine 
ulaşmadıysa yeni uçağın indirilmesi ve eski uçağın ertelenmesi için "yui()" 
fonksiyonu çağırılır. Eğer eski uçak azami gecikme süresine ulaştıysa yeni 
uçağın ertelenmesi için "yue()" fonksiyonu çağırılır. İkinci durumda ise yeni 
uçağın ertelenmesi için "yue()" fonksiyonu çağırılır. Üçüncü durumda ise iniş 
saati çakışan uçakların önceliği uçakların ID'sine göre karar verilir. ID sayısı
düşük olan uçak daha önceliklidir. İlk iki durumdaki gibi gerekli kontroller 
yapılır ve gerekli fonksiyonlar çağırılır.\\

\subsection{"yue()" Fonksiyonu}
Yeni eklenecek uçakların ertelenmesi için oluşturulan bu fonksiyonda eklenecek 
uçak verisi en fazla 3 kere tekrar edebilecek bir döngüye girer. Her tekrarda 
kuyruktaki uçağın verisiyle yeni uçağın verileri karşılaştırılır. Karşılaştırma 
yapılırken öncelikle kuyruktaki mevcut uçağın azami gecikme süresine ulaşıp 
ulaşmadığı kontrol edilir. Eğer uygun bir durum oluşmazsa yeni uçağa iniş izni 
verilemez ve Sabiha Gökçen Havalimanı'na yönlendirilir.\\

Uygun bir durum oluştuğunda ise "arayaEkle()" fonksiyonu ile yeni uçak gecikme ve 
yeni iniş saati bilgileriyle beraber kuyruktaki yerine eklenir ve eski uçağın 
bilgileri oluşturulan geçici değişkenlere atandıktan sonra ertelenebilmesi için 
"ee()" fonksiyonu çağırılır.\\

\begin{figure}[htbp]
\includegraphics[width=8cm, height=17cm]{akissema2.png}
\caption{Yeni Uçağın Ertelenmesi Akış Şeması}
\end{figure}

\subsection{"yui()" Fonksiyonu}
Yeni uçağın gecikmesiz bir şekilde kuyruğa eklenebilmesi için oluşturulan bu 
fonksiyonda yeni uçak "arayaEkle()" fonksiyonu ile kuyruğa eklenir. Kuyrukta 
bulunan mevcut uçağın bilgileri oluşturulan geçici değişkenlere atandıktan sonra
"aradanSil()" fonksiyonu ile kuyruktan silinir ve ertelenmesi için "ee()" 
fonksiyonu çağırılır.\\

\subsection{"ee()" Fonksiyonu}
Halihazırda kuyrukta bulunan uçakların ertelenebilmesi için oluşturulan bu 
fonksiyonda öncelikle uçağın gecikme değeri bir arttırılır. Yeni gecikme değeri 
3'e ulaştıysa yeni iniş saatinin olduğu saatte pistin dolu olup olmadığı kontrol
edilir. Eğer boşsa eski uçağın iniş bilgileri yeni gecikme ve iniş saati 
bilgileriyle "arayaEkle()" fonksiyonu ile kuyruğa tekrar eklenir. Eğer doluysa 
da kuyruktaki iniş saati çakışan uçağın azami gecikme süresine ulaşıp ulaşmadığı
kontrol edilir. Ulaşmadıysa eski uçak, diğer eski uçağın yerine geçer ve diğer 
eski uçağın ertelenmesi için "ee()" fonksiyonu tekrardan çağırılır. Eğer uçak 
azami gecikme süresine ulaştıysa ertelenmek istenen eski uçağın izni iptal 
edilir.\\

Mevcutta bulunan uçak ertelenirken gecikme değeri arttırıldıktan sonra 3'e 
ulaşmamışsa uçak bir döngüye girer. Bu döngüde kuyrukta uygun yer bulma ve 
erteleme işlemleri yapılır. Uçak boş saat bulabilmişse kuyruktaki uygun boşluğa 
eklenir. Eğer kuyruktaki başka bir uçağın yerini alırsa, kuyruktaki uçağın da 
ertelenmesi için "ee()" fonksiyonu tekrar çağırılır.\\

\subsection{Kuyruk Kapasitesinin Dolması Durumu}
Yeni bir uçak eklenirken kuyruk kapasitesi dolmuşsa, kuyrukta "Önceliği yeni 
uçaktan düşük ve iniş saati yeni uçağın iniş saatine eşit veya daha geç saatte 
olan uçak var mı?" diye kontrol edilir. Eğer varsa yeni uçak eski uçağın yerini 
alır ve eski uçağın iniş izni iptal edilir. Eğer yoksa yeni uçağa iniş izni 
verilemez ve Sabiha Gökçen Havalimanı'na yönlendirilir.\\

\subsection{"guncel()" Fonksiyonu}
"input.txt" dosyasından ilk uçak bilgileri okunduktan sonra her bir okuma öncesi
ve tüm okumalar bittikten sonra "guncel()" fonksiyonu çağırılır. Bu fonksiyonda 
"kalkis\_pisti\_kullanim\_sirasi" kuyuruğundaki tüm elemanlar silinir. Ardından 
"inis\_pisti\_kullanim\_sirasi" kuyruğundaki ilk ve son düğümler hariç tüm 
düğümler "kalkis\_pisti\_kullanim\_sirasi" öncelikli kuyruğuna da eklenir.\\

"output.txt" dosyası yazma modunda açılır. Bu dosyanın ilk satırına "oncelik\_id
ucak\_id talep\_edilen\_inis\_saati inis\_saati gecikme\_suresi kalkis\_saati" 
kelime grubu yazdırılır. Sonrasında "kalkis\_pisti\_kullanim\_sirasi" öncelikli 
kuyruğundaki düğümlerde saklanan uçak verileri "output.txt" dosyasına 
yazdırılır ve dosya kapatılır. "k\_yazdir()" fonksiyonu da çağırılarak 
"kalkis\_pisti\_kullanim\_sirasi" öncelikli kuyruğundaki veriler ekrana 
yazdırılır.\\

\section{Deneysel Sonuçlar}
Öncelikli kuyrukların temel fonksiyonları geliştirilerek proje amacına uygun 
olacak şekilde düzenlendi.\\

Yeni uçakların gelmesi ve eski uçakların ertelenmesinden dolayı koşul 
kontrolleri sonsuz döngüye giriyordu. Koşul kontrollerini fonksiyon içerisinde 
yapıp, o fonksiyonların içerisinde de aynı fonksiyonu tekrardan çağırarak bu 
sonsuz döngüden çıkıldı.\\

Azami gecikme süresine ulaşan uçaklar tekrardan ertelenmeye çalışıldığında, 
eski uçak kuyruktan siliniyor ve uçak verisi kayboluyordu. Azami gecikme 
süresine ulaşan uçakları en yüksek öncelikli uçak kategorisine uygun 
kontrollerden geçirerek 3'ten fazla ertelenmemesi sağlandı.\\

~\\
~\\
~\\
~\\
~\\
~\\
~\\
~\\
~\\
~\\
~\\
~\\
~\\
~\\
~\\
~\\
~\\
~\\
~\\

\section{Sonuç}
Bağlı listeler ve öncelikli kuyruklar kullanılarak "Havalimanı Uçuş Yönetim 
Sistemi" hazırlandı. Alınan veriler belirli koşullara ve önceliklerine göre 
sıralandı. Çeşitli fonksiyonlar kullanılarak karışıklığın azaltılması sağlandı.\\

\renewcommand{\refname}{Kaynakça}
\begin{thebibliography}{00}
\bibitem{b1} medium.com. (2019). Doğrusal Veri Yapılar 4, Erişim Tarihi: 10.12.2021, https://medium.com/@tolgahan.cepel/dogrusal-veri-yapilari-4-kuyruk-queue-dcbd07e8ba77/
\bibitem{b2} cdn-acikogretim.istanbul.edu.tr. KUYRUK (QUEUE) VERİ YAPISI, Erişim Tarihi: 10.12.2021, https://cdn-acikogretim.istanbul.edu.tr/auzefcontent/20\_21\_Guz/veri\_yapilari/7/
\bibitem{b3} yazilimtuneli.com. (2019). C Programlama Yığın ve Kuyruk Örnek, Erişim Tarihi: 10.12.2021, http://www.yazilimtuneli.com/2019/11/c-programlama-ygn-ve-kuyruk-ornek/
\bibitem{b4} Yazılıma Gönül Ver. (2019). C++ Priority Queue ( Öncelikli Kuyruk ) Sınıfı ve Methodları, Erişim Tarihi: 10.12.2021, https://www.youtube.com/watch?v=IqYh57D67-g/
\bibitem{b5} diagrams.net. Akış Şeması, Erişim Tarihi: 12.12.2021, https://app.diagrams.net/
\end{thebibliography}
\end{document}
